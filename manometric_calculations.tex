%\documentclass[twocolumn]{article}
\documentclass[]{article}
\usepackage[version=3]{mhchem}
\usepackage{sectsty}
\usepackage{amsmath}
\usepackage[flushleft]{threeparttablex} % For table notes
\usepackage{rotating} 
\usepackage{longtable}
\usepackage{hyperref}
% For decimal alignment in tables
\usepackage{dcolumn}

% For dcolumn
\newcolumntype{.}{D{.}{.}{3}}

\sectionfont{\large}

\newcommand{\unit}[1]{\ensuremath{\, \mathrm{#1}}}

\title {Calculation of methane production from manometric measurements}
\author{Sasha D. Hafner, Sergi Astals, and Pierre Buffiere\\
\\
\texttt{sasha@hafnerconsulting.com} (S. D. Hafner)\\
\texttt{sastals@ub.edu} (S. Astals)\\
} 

\begin{document}
\maketitle

\section{BMP-methods}
File version 2.2. 
This file is from the GitHub repository BMP-methods.
For more information, visit BMP-methods at \url{https://github.com/sashahafner/BMP-methods}.

\section{Description}
This document describes calculations for manometric measurment of biogas.
Two methods are commonly used and both are described here: one based on normalized \ce{CH4} concentrations (method 1) and one that explicitly includes estimation of \ce{CH4} in the bottle headspace (method 2).
Expected results from the two methods are identical; differences are due only to error in measurement of biogas composition or headspace volume.
Both methods are available through the \texttt{cumBg()} function in the biogas package \cite{softwarex} and through the web application OBA (\url{https://biotransformers.shinyapps.io/oba1/}) and can be easily added to, e.g., a spreadsheet template.

\section{Standardization of measured gas volume}
Both methods use the same approach for standardization of gas volume.
Dry biogas volume in a bottle's headspace before and after venting is calculated by correcting for water vapor, temperature, and pressure.
First the volume of headspace gas is converted to dry conditions at standard pressure:

\begin{equation}
  \label{eq:h2ocorrection}
  V_{dry} = V_{headspace}(P_{meas} - P_{H_2O})/101.325 \unit{kPa}
\end{equation}
where $P_{meas}$ is the measured headspace pressure and $P_{H_2O}$ the water vapor partial pressure (both in kPa).
Eq. (\ref{eq:h2ocorrection}) is an expression of Boyle's law.
The value of $P_{H_2O}$ is assumed to be the saturation vapor pressure prior to venting, and can be calculated using, e.g., the Magnus-form equation given below (Eq. 21 in \cite{magnus}):

\begin{equation}
\label{eq:magnus}
   P_{H_2O} = 0.61094 e^{(17.625 T/(243.04 + T))}
\end{equation}
where T is temperature in $^\circ$C.
Immediately after venting, when water vapor has been lost through venting but water evaporation has not yet led to equilibrium, it may be reasonable to assume that the mixing ratio of water is the same as it was prior to venting.
Assuming saturation prior to venting and an insignificant change in temperature, this is equivalent to taking relative humidity as the ratio of post- to pre-venting absolute pressure (always $<1.0$). 
Alternatively, it may be no less accurate to assume 100\% relative humidity after venting as well, especially if more than a few minutes pass between venting and measurement of residual pressure.
Althought the difference in resulting BMP estimates is not large (perhaps a few \%), it is prudent to state which of these two assumptions was made when reporting BMP results.

Volume is then further standardized to 273.15 K by application of Charles's law:

\begin{equation}
  \label{eq:bgstd}
  V_{std} = V_{dry} 273.15 \unit{K}/T_{meas}
\end{equation}
where $V_{std}$ is the standardized volume of gas within a bottle's headspace at the time of pressure measurement.

Interval biogas production is calculated as:

\begin{equation}
  \label{eq:bgint}
  V_{biogas, i} = V_{std, pre, i} - V_{std, post, i - 1}
\end{equation}
where all $V$ is standardized gas volume in a bottle's headspace, $pre$ and $post$ refer to before and after venting, respectively, $i$ indicates the current interval and $i-1$ the previous one.

\section{Calculation of \ce{CH4} production}
\subsection{Method 1}
In the first method, biogas is assumed to consist of only \ce{CH4} and \ce{CO2} at the time of production (i.e., as produced by the microbial community) and \ce{CH4} production is calculated from vented (removed) biogas only.
This method is described in \cite{brian1991}.
Coupled with the assumption that all gas production is biogas (Eq. \ref{eq:bgint}), this provides the simplest approach for calculating \ce{CH4} production.

First, concentrations of \ce{CH4} and \ce{CO2} are adjusted so they sum to 1.0:
\begin{equation}
  x_{CH_4, n} = x_{CH_4}/(x_{CH_4} + x_{CO_2})
\end{equation}
where $x_{CH_4}$ and $x_{CO_2}$ are the measured \ce{CH4} and \ce{CO2} concentrations as volume (mole) fraction (possibly including a correction for water vapor--this has no effect here) and $x_{CH_4, n}$ is the normalized \ce{CH4} volume fraction.

Methane production in an interval $i$ is then calculated as
\begin{equation}
  \label{eq:ch4m1}
  V_{CH_4, i} = x_{CH_4, n} V_{biogas, i}
\end{equation}
Cumulative production is taken as the cumulative sum of interval values. 

\subsection{Method 2}
Method 2 relies on fewer assumptions, but requires the true concentration of \ce{CH4} (volume fraction) of \ce{CH4} within the bottle headspace, with correction only for water vapor.
Here, \ce{CH4} production in an interval is determined from the change in bottle \ce{CH4} content during an incubation interval.
\begin{equation}
  \label{eq:ch4m2}
  V_{CH_4, i} = x_{CH_4, i} \cdot V_{std, pre, i} - x_{CH_4, i - 1} \cdot V_{std, post, i - 1}
\end{equation}
Cumulative production is taken as the cumulative sum of interval values. 
Note that Eq. (\ref{eq:ch4m2}) reduces to Eq. (\ref{eq:ch4m1}) when $x_{CH_4, n} = x_{CH_4}$ and $x_{CH_4, i} = x_{CH_4, i - 1}$ , i.e., if all flusing gas has been removed from the bottle headspace, biogas production is indeed only \ce{CH4} and \ce{CO2}, and biogas composition has not changed during an interval.

\begin{thebibliography}{1}

\bibitem{softwarex}
Hafner, S.D., Koch, K., Carrere, H., Astals, S., Weinrich, S., Rennuit, C.
    \newblock{2018}
    \newblock{Software for biogas research: Tools for measurement and prediction of methane production}. 
    \newblock{SoftwareX} 7: 205-210

\bibitem{magnus}
Alduchov, O.A., Eskridge, R.E.    
    \newblock{1996}
    \newblock{Improved Magnus form approximation of saturation vapor pressure}. 
    \newblock{Journal of Applied Meteorology} 35: 601-609

\bibitem{brian1991}
Richards, B.K., Cummings, R.J., White, T.E., Jewell, W.J.    
    \newblock{1991}
    \newblock{Methods for kinetic-analysis of methane fermentation in high solids biomass digesters}. 
    \newblock{Biomass and Bioenergy} 1: 65-73

\end{thebibliography}

\bibliographystyle{plain}
%\bibliography{gbib}

\end{document}
