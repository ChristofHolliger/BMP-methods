%\documentclass[twocolumn]{article}
\documentclass[]{article}
\usepackage[version=3]{mhchem}
\usepackage{sectsty}
\usepackage{amsmath}
\usepackage[flushleft]{threeparttablex} % For table notes
\usepackage{rotating} 
\usepackage{longtable}
\usepackage{hyperref}
% For decimal alignment in tables
\usepackage{dcolumn}

% For dcolumn
\newcolumntype{.}{D{.}{.}{3}}

\sectionfont{\large}

\newcommand{\unit}[1]{\ensuremath{\, \mathrm{#1}}}

\title {Calculation of methane production from volumetric measurements}
\author{Sasha D. Hafner\\
\\
\texttt{sasha.hafner@eng.au.dk}
} 

\begin{document}
\maketitle

\section{BMP-methods}
File version 1.2. 
This file is from the GitHub repository BMP-methods.
For more information, visit BMP-methods at \url{https://github.com/sashahafner/BMP-methods}.

\section{Description}
This document describes calculations for volumetric measurment of biogas.
As with manometric methods, two methods are commonly used and both are described here: one based on normalized \ce{CH4} concentrations (method 1) and one that explicitly includes estimation of \ce{CH4} in the bottle headspace (method 2).
Expected results from the two methods are identical; differences are due only to error in measurement of biogas composition or headspace volume.
Both methods are available through the \texttt{cumBg()} function in the biogas package \cite{softwarex} and through the web application OBA (\url{https://biotransformers.shinyapps.io/oba1/}) and can be easily added to, e.g., a spreadsheet template.

\section{Standardization of measured gas volume}
Both methods use the same approach for standardization of gas volume.
Dry biogas volume in a bottle's headspace before and after venting is calculated by correcting for water vapor, temperature, and pressure.
First the measured gas volume (e.g., in a syringe or hanging water column) is converted to dry conditions at standard pressure:

\begin{equation}
  \label{eq:h2ocorrection}
  V_{dry} = V_{headspace}(P_{meas} - P_{H_2O})/101.325 \unit{kPa}
\end{equation}
where $P_{meas}$ is the measured headspace pressure and $P_{H_2O}$ the water vapor partial pressure (both in kPa).
Eq. (\ref{eq:h2ocorrection}) is an expression of Boyle's law.
The value of $P_{H_2O}$ is assumed to be the saturation vapor pressure, and can be calculated using, e.g., the Magnus-form equation given below (Eq. 21 in \cite{magnus}):

\begin{equation}
\label{eq:magnus}
   P_{H_2O} = 0.61094 e^{(17.625 T/(243.04 + T))}
\end{equation}
where T is temperature in $^\circ$C.
Volume is then further standardized to 273.15 K by application of Charles's law:

\begin{equation}
  \label{eq:bgstd}
  V_{std} = V_{dry} 273.15 \unit{K}/T_{meas}
\end{equation}
where $V_{std}$ is the standardized volume of gas within a bottle's headspace at the time of pressure measurement.
Interval biogas production $V_{biogas, i}$ is taken as this standardized volume $v_{std}$.
Cumulative production is taken as the cumulative sum of interval values. 

\section{Calculation of \ce{CH4} production}
\subsection{Method 1}
In the first method, biogas is assumed to consist of only \ce{CH4} and \ce{CO2} at the time of production (i.e., as produced by the microbial community) and \ce{CH4} production is calculated from vented (removed) biogas only.
This method is described in \cite{brian1991}.
Coupled with the assumption that all gas production is biogas, this provides the simplest approach for calculating \ce{CH4} production.

First, concentrations of \ce{CH4} and \ce{CO2} are adjusted so they sum to 1.0:
\begin{equation}
  x_{CH_4, n} = x_{CH_4}/(x_{CH_4} + x_{CO_2})
\end{equation}
where $x_{CH_4}$ and $x_{CO_2}$ are the measured \ce{CH4} and \ce{CO2} concentrations as volume (mole) fraction (possibly including a correction for water vapor--this has no effect here) and $x_{CH_4, n}$ is the normalized \ce{CH4} volume fraction.

Methane production in an interval $i$ is then calculated as
\begin{equation}
  V_{CH_4, i} = x_{CH_4, n} V_{biogas, i}
\end{equation}
Cumulative production is taken as the cumulative sum of interval values. 

\subsection{Method 2}
Method 2 relies on fewer assumptions, but requires the true concentration of \ce{CH4} (volume fraction) of \ce{CH4} within the bottle headspace, with correction only for water vapor.
Here, \ce{CH4} production in an interval has two components: a vented part that is naturally interval, and a residual headspace part, that is naturally cumulative:
\begin{equation}
  V_{CH_4, i} = V_{CH_4, v, i} + ( V_{CH_4, HSR, i} - V_{CH_4, HSR, i-1} )
\end{equation}
where the subscript $v$ indicates vented volume and $HSR$ = residual headspace volume (post-venting).

Vented \ce{CH4} is calculated from:
\begin{equation}
  V_{CH_4, v, i} = x_{CH_4, n, i} V_{biogas, i}
\end{equation}

Headspace \ce{CH4} is calculated from:
\begin{equation}
  V_{CH_4, HSR, i} = x_{CH_4, n, i} V_{post, i}
\end{equation}
where $V_{post}$ is the post-venting standardized volume of gas in the bottle headspace.

Cumulative production is taken as the cumulative sum of interval values. 

\section{Example Calculations}
In the following example, \ce{CH4} production is calculated from a single interval measurement made on a single bottle in a BMP trial. Calculations are made using both volumetric method 1 and 2. 

For both methods standardized gas volume is calculated from Eq. (\ref{eq:bgstd}) by correcting for water vapor, temperature, and pressure. Measured biogas volume ($V_{biogas,i}$) was 618 mL at a temperature ($T_{meas}$) of 20\degree C and atmospheric pressure ($P_{meas}$) of 101.325 kPa.  
First water vapor pressure is calculated at the measured headspace temperature using Eq (\ref{eq:magnus}).
\begin{equation*}
   P_{H_2O} = 0.61094 \cdot e^{\frac{17.625 \cdot 20 \degree C}{243.04 + 20 \degree C}} = 2.333\ kPa
\end{equation*}
Secondly, the headspace volume is converted to dry conditions at standard pressure using Eq. (\ref{eq:h2ocorrection}).
\begin{equation*}
   V_{dry} = \frac{618\ mL \cdot (101.325\ kPa - 2.333\ kPa)}{101.325\ kPa} = 592.158\ mL  
\end{equation*}
Then, volume is further standardized following Eq. (\ref{eq:bgstd}).
\begin{equation*}
    V_{std} = \frac{592.158\ mL \cdot 273.15\ K}{293.15\ K} = 551.758\ mL  
\end{equation*}
$V_{biogas,i}$ is taken as this $V_{std}$. Cumulative production is taken as the cumulative sum of interval values.
\subsection{Method 1}
The mole fraction of $CH_{4}$ ($x_{CH_4}$, dimensionless) normalized for $CH_{4}$ and $CO_{2}$ can be calculated according Eq. (4), but was given as 0.627. 
Then, following Eq. (5), $CH_{4}$ production in the interval is calculated from interval biogas production.
\begin{equation*}
  V_{CH_4, i} = 0.672 \cdot 551.758\ mL  = 345.952\ mL 
\end{equation*}
\subsection{Method 2}
Post-venting dry and standardized volume for current interval (\textit{i}) and the previous interval (\textit{i-1}) is required for method 2 and can be calculated following Eq. (\ref{eq:h2ocorrection}) and (\ref{eq:bgstd}), respectively. Post headspace pressure ($P_{post}$) was assumed to be constant at 1.01 bar throughout the BMP trial. 

Assuming constant biogas composition, $CH_{4}$ production in the interval can be calculated following Eq. (6). 
Vented $CH_{4}$ volume ($V_{CH_4, v,i}$) is obtained from the interval biogas production and the mole fraction of $CH_{4}$ using Eq. (7).
\begin{equation*}
  V_{CH_4, v, i} = 0.672 \cdot 551.758\ mL  = 345.952\ mL 
\end{equation*}
Under the assumptions of constant post headspace pressure and gas composition,  Eq. (6) reduces to Eq. (7). Meaning that $CH_{4}$ production in the interval is equal to vented $CH_{4}$ volume and hence, method 2 equals method 1. 
\begin{equation*}
  V_{CH_4, i} = V_{CH_4, v, i} = 345.952\ mL    
\end{equation*}


\begin{thebibliography}{1}

\bibitem{softwarex}
Hafner, S.D., Koch, K., Carrere, H., Astals, S., Weinrich, S., Rennuit, C.
    \newblock{2018}
    \newblock{Software for biogas research: Tools for measurement and prediction of methane production}. 
    \newblock{SoftwareX} 7: 205-210

\bibitem{magnus}
Alduchov, O.A., Eskridge, R.E.    
    \newblock{1996}
    \newblock{Improved Magnus form approximation of saturation vapor pressure}. 
    \newblock{Journal of Applied Meteorology} 35: 601-609

\bibitem{brian1991}
Richards, B.K., Cummings, R.J., White, T.E., Jewell, W.J.    
    \newblock{1991}
    \newblock{Methods for kinetic-analysis of methane fermentation in high solids biomass digesters}. 
    \newblock{Biomass and Bioenergy} 1: 65-73

\end{thebibliography}

\bibliographystyle{plain}
%\bibliography{gbib}

\end{document}
