

To measure the BMP of a substrate, a small sample is mixed with an anaerobic inoculum in bottles, which are then sealed and incubated. 
Methane \ce{CH4} production is measured over time.
Production of \ce{CH4} from the inoculum is estimated by making similar measurements on bottles with only inoculum (``blanks'').
For more details on the basic approach, see Owen et al. \cite{owen1979}.

\subsection{Inoculum}
\begin{enumerate}
  \item Origin. 
    A highly diverse methanogenic microbial community should be present.
    May be taken from mesophilic (35-40$^\circ$C) or thermophilic ($> 40^\circ$C) digesters. 
    Recommended sources are digestate samples from sludge digesters at wastewater treatment plants or digesters treating mixed agricultural waste.
    Mixing samples from multiple sources is acceptable.
  \item Treatment. 
    Sieving or other pre-treatment (grinding, dilution) are acceptable if needed (sieving to remove very coarse materials (any dimension $> 10$ mm), dilution if total solids (TS) $> 10\%$), but generally should be avoided.
  \item Analysis.
    \begin{enumerate}
      \item Total solids (TS) by drying for at least 24 hours at 105$^\circ$C in triplicate.
      \item Volatile solids (VS) by combusting at 550$^\circ$C for at least 2 hours in triplicate.
    \end{enumerate}
  \item Quality check before use.\\
    Required checks:
    \begin{enumerate}
      \item pH must be between 7.0 and 8.5.
    \end{enumerate}

    Recommended checks:
    \begin{enumerate}
      \item Alkalinity should be $\ge$ 3 g L$^{-1}$ as \ce{CaCO3}.
      \item Total ammonical nitrogen (TAN) should be $<$ 2.5 g L$^{-1}$ as N
      \item Total volatile fatty acids (VFAs) should be $<$ 1.0 g L$^{-1}$ as acetic acid
    \end{enumerate}
  \item Storage. Storage time between collection and setting up BMP tests $\le$ 5 days at ambient (20-25$^\circ$C) or test temperature.
  \item Methane productivity.
    It is recommended that inoculum should contribute $<$ 40\% of the gross methane production of the positive control. 
    A recommended rule-of-thumb is that initial \ce{CH4} production from inoculum is $< 20$ mL g$^{-1}$ d$^{-1}$ (VS basis).
    More active inoculum can be stored at test temperature for up to 5 days to reduce activity.

\end{enumerate}


  \item Substrate quantity and bottle size. 
    At least 1.0 g of substrate VS must be added to each bottle. 
    This affects bottle size, and care should be taken to avoid high pressure in manual methods (due to leaks or bottle breakage). 
    Recommended maximum headspace pressure is 2 bar (gauge). 
    A recommended rule-of-thumb is 3-10 g substrate VS per L headspace volume, as long as daily sampling is possible at the start. 
    Total bottle volume is recommended to be 500 - 1000 mL.
  \item Inoculum-to-substrate ratio (ISR). On a VS basis, ISR should generally be 2, but may be as low as 1 for slowly degradable substrates, and as high as 4 for easily degradable substrates.
  \item Amendments. 
    \begin{enumerate}
      \item Trace element and vitamin amendment is required.\footnote{
          Trace element solution (concentration in g L$^{-1}$): 2 \ce{FeCl2\cdot4H2O}, 0.05 \ce{H3BO3}, 0.05 \ce{ZnCl2}, 0.038 \ce{CuCl2\cdot2H2O}, 0.05 \ce{MnCl2\cdot4H2O}, 
          0.05 \ce{(NH4)6Mo7O24\cdot4H2O}, 0.05 \ce{AlCl3}, 0.05 \ce{CoCl2\cdot6H2O}, 0.092 \ce{NiCl2\cdot6H2O}, 0.5 ethylenediaminetetraacetate, 1 mL concentrated HCl, 
          0.1 \ce{Na2SeO3\cdot5H2O}.
          \newline
          Vitamin mixture (concentration in mg L$^{-1}$): 2 Biotin, 2 folic acid, 10 pyridoxine acid, 5 riboflavin, 5 thiamine hydrochloride, 0.1 cyanocobalamine, 
          5 nicotinic acid, 5 P-aminobenzoic acid, 5 lipoic acid, X???? DL-pantothenic acid.
          \newline
          Add between 1 and 5 mL of each solution per 1 L of final slurry volume (typically 1 mL each per bottle).
        }.
      \item If alkalinity is too low, add \ce{NaHCO3} to meet requirement.
    \end{enumerate}
  \item Headspace flushing. Flush headspace prior to incubation to remove \ce{O2}. 
    Use a mixture of \ce{N2} and \ce{CO2} (20-40\% \ce{CO2}) or 100\% \ce{N2}. Do not flush liquid phase with pure \ce{N2}. 
    It is recommended to measure gas flow rate and to replace at least 3 headspace volumes.
  \item Incubation.
    \begin{enumerate}
      \item Temperature controlled environment at mesophilic temperature (35-40$^\circ$C) with $\le$ 2$^\circ$C variation during incubation. 
        The temperature should match the temperature of the digester that was the source of inoculum.
      \item Mixing is compulsory, if manually at least once a day.
    \end{enumerate}

  \item Methane production measurement.
    \begin{enumerate}
      \item No restrictions on which system to use.
      \item The system should be designed to be gas-tight and should be tested for leakage. 
        Measurement of leakage during BMP tests is recommended if at all possible.
        The approach described by Hafner et al. \cite{leaks2018} is compatible with most manual BMP methods.
        Detection of major leakage, defined here as $> 20\%$ of total biogas volume for any single bottle, is not acceptable, and in that case results should be discarded.
        Smaller leaks can be corrected if quantified \cite{leaks2018}.
        Results should be discarded if leakage is detected but not quantified.
      \item If gas composition has to be analyzed, it has to be analyzed at each measuring point and for every single batch (bottle).
      \item At each measuring point, ambient pressure and temperature has to be measured and recorded for use in gas volume standardization.
    \end{enumerate}



\bibitem{bmpprotocol}
Holliger, C., . . .
    \newblock{2020},
    \newblock{General protocol for measurement of biochemical methane potential (BMP)},
    \newblock{\url{https://github.com/sashahafner/BMP-methods}}


\bibitem{owen1979}
Owen, W. F., Stuckey, D. C., Healy Jr, J. B., Young, L. Y., McCarty, P. L.
\newblock{1979},
\newblock{Bioassay for monitoring biochemical methane potential and anaerobic toxicity}
\newblock{Water Research} 13: 485-492.

\bibitem{leaks2018}
Hafner, S.D., Rennuit, C., Olsen, P.J., Pedersen, J.M.
\newblock{2018},
\newblock{Quantification of leakage in batch biogas assays} 
\newblock{Water Practice and Technology} 13, 52–61
    \newblock{\url{https://doi.org/10.2166/wpt.2018.012}}

