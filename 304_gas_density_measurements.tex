\documentclass[]{article}
\usepackage[version=3]{mhchem}
\usepackage{sectsty}
\usepackage{amsmath}
\usepackage[flushleft]{threeparttablex} % For table notes
\usepackage{rotating} 
\usepackage{longtable}
\usepackage{hyperref}
% For decimal alignment in tables
\usepackage{dcolumn}

% For dcolumn
\newcolumntype{.}{D{.}{.}{3}}

\newcommand{\unit}[1]{\ensuremath{\, \mathrm{#1}}}

\title {Gas density-based (GD) BMP measurement}
\author{Sasha D. Hafner, Jacob R. Mortensen, Sergi Astals\\
\\
\texttt{sasha.hafner@eng.au.dk, sasha@hafnerconsulting.com} (S. D. Hafner)
} 

\begin{document}

\maketitle

\section{BMP-methods}
File version 1.2. 
This file is from the GitHub repository BMP-methods.
For more information, visit BMP-methods at \url{https://github.com/sashahafner/BMP-methods}.

\section{Overview}
By measuring BMP bottle mass loss and biogas volume, gas density can be calculated, and from this, biogas composition.
This document describes the laboratory measurements needed for applying this "GD-BMP" approach.

\section{Protocol}

\subsection{Required equipment and supplies}

\begin{itemize}
    \item Digital scale
    \item Syringes and needles
    \item Manometer
    \item Typical BMP bottles and septa
\end{itemize}

The required accuracy of the scale will depend on the quantity of biogas produced. 
For 1 g of substrate VS, readability should be 0.01 g (10 mg) or better (with accuracy no more than 3 times as large).
However, accuracy as large as 0.2 g may be sufficient, but is less than ideal.

A simple closed u-tube manometer is sufficient for determining that post-venting headspace pressure is close to atmospheric.
One can be made with some plastic tubing filled with water and a simple valve (made, e.g., by folding flexible tubing).

It is best to have several sizes of syringes, in order to have high relative precision in volume measurements when biogas production is both high or low.
Ideally the largest syringe will be large enough to measure the largest volume of biogas produced in a single interval (e.g., 1 L if the largest volume is 800 mL of biogas).
But large syringes are expensive, and not necessary.
Instead, a single small syringe can be used multiple times to remove the biogas from a single bottle in a single interval.
But this approach requires that the manometer is directly connected to the syringe and there is a valve between the syringe and the bottle.
Videos showing this approach will be available in the near future.

\subsection{Setup}
During setup, inoculum and substrate are added to bottles, and the headspace of each bottle is flushed to remove \ce{O2} and ensure anaerobic conditions. 
Pure \ce{N2} is preferred for flushing over mixtures containing \ce{CO2}\footnote{
Flushing gas results in a (generally small) error because its density may differ from produced biogas density (the density of \ce{N2} is identical to a \ce{CH4}:\ce{CO2} mixture with 58\% \ce{CH4}, and higher for a mixture with more \ce{CH4}) but this can be corrected in calculations. 
}.
Bottles are then weighed and placed in an incubator.

\subsubsection{Step-by-step instructions}
\begin{enumerate}
    \item Check the accuracy of the scale with a weight set. 
      It is particularly important that the actual accuracy is close to reported accuracy when weighing an object with a mass close to the total mass of a BMP bottle and its contents. 
      For a scale with a reported accuracy of 50 mg, for example, this could be checked by taring the scale with a full bottle or equivalent mass, and adding a 50 mg weight.
    \item Add the required mass of inoculum, substrate, and other additions (e.g., a trace element solution) to each labeled bottle and seal with a septum and cover. 
      Determination of the quantity of material added by mass difference is the recommended approach: Tare scale with bottle, add approximately the desired quantity, wipe any material from near the mouth of the bottle, and finally determine the actual quantity from the scale reading. 
      Note that the scale used here does not need to be the same scale used for determining mass loss (see ``Incubation and sampling'', below).
    \item Flush the bottle headspace to remove \ce{O2}. 
      A simple approach is to use a needle attached to a flow meter (e.g., a rotameter), a pressure regulator (to ensure low pressure), and a gas cylinder (generally with \ce{N2}) with plastic tubing, along with a separate needle for venting. 
      Minimize \ce{CO2} removal by flushing for only 3 to 4 headspace volume exchanges. 
      Ensure that the flushing gas does not bubble through the liquid in the bottle (needle should not be submerged). 
      Allow the pressure in each bottle’s headspace to equilibrate with atmospheric pressure before removing the venting needle.
    \item Make 2 ``water control'' bottles that contains only water. 
      They should be the same size and weigh about as much as the other BMP bottles. 
      These bottles should never be vented; they are used to check the stability of the scale and it is essential that they do not lose any mass.
    \item Weigh each bottle and record as ``initial mass''. 
      Repeat this initial weighing in order to minimize the chance of a recording error, because calculations of cumulative \ce{CH4} production at all timepoints require an accurate initial mass measurement.
      If there is a discrepancy between these two initial measurements, weigh again to determine the correct mass.
      It is important that the only change in bottle mass after this time is due to biogas removal.
      Bottles should be kept clean, and labels should not be added after this time, for example.
    \item Place bottles in incubator set at the test temperature.
\end{enumerate}

\subsection{Incubation and sampling}
Bottles are removed from the incubator occasionally to vent and weigh in what is here referred to as a ``sampling event''. 
Biogas temperature affects water vapor content. 
To minimize uncertainty in the headspace temperature used in calculations, the time that bottles spend outside the incubator should be short, and the same procedure and timing should be followed for each sampling event. 
(Ideally, venting and weighing should be done inside a temperature-controlled room, so bottles are always at the incubation temperature. 
However, the effects of headspace temperature on accuracy are small, so this is not required.)

The accuracy of the GD-BMP method is only slightly affected by headspace pressure, and it is possible to correct for leakage of biogas. 
However, for safety (to avoid exploding bottles), for maximum precision, and to minimize possible effects of high \ce{CO2} dissolution, total headspace pressure (absolute) should be kept below 3 bar (2 bar gauge pressure). 
Bottle pressures can be estimated from headspace and vented biogas volume.

\subsubsection{Step-by-step instructions}
\begin{enumerate}
    \item Remove both water control bottles from the incubator and weigh them to confirm scale consistency. 
      If the results are the same as the initial masses (within the expected accuracy) proceed, otherwise, identify and address the problem with the scale or replace the scale if necessary.
      If the problem cannot be resolved, proceed and later correct mass results for scale drift.
    \item Remove a single set of replicates from the incubator (e.g., the three replicates for cellulose).
    \item Always starting with the same replicate (e.g., ``1'' or ``a'') gently swirl the bottle for at least 10 s to mix the contents and encourage \ce{CO2} equilibration between solution and headspace. 
      During swirling, avoid contact between the liquid and the septum.\footnote{
        If the septum becomes contaminated with reacting material, a small amount may be pushed out during venting, which will result in error in the determination of mass loss.}
    \item Weigh the bottle and record the result as pre-venting mass.
    \item Vent the bottle using a syringe and measure biogas volume.
      Use the manometer to ensure that the headspace pressure is close to atmospheric after venting (gauge pressure $\pm3$ kPa).
      Note any loss of reacting material through the needle or other unusual event.
    \item Weigh the bottle after venting, and record the result as post-venting mass. 
    \item Proceed to the next replicate (e.g., ``2'' or ``b'') and repeat steps 3 - 6.
    \item After all replicates have been mixed, weighed, vented, and weighed again, place the bottles back in the incubator.
    \item Proceed to the next set of replicates (e.g., the three replicates for substrate ``food waste A'') and repeat steps 2 - 9.
\end{enumerate}

\end{document}
